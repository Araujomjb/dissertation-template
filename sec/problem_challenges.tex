\chapter{The problem and its challenges}
         The problem and its challenges.

\section{Proposed Approach - solution}
In this section, it is presented various ways to display an image.
   \subsection{System Architecture}
   A block diagram of the planned system / approach

Here we have an example of inserting an image between the text paragraphs.
\begin{center}
  \includegraphics[width=0.2\textwidth]{img/mei-logo-03.jpg}
\end{center}

\begin{wrapfigure}{r}{0.25\textwidth}
  \includegraphics[width=0.2\textwidth]{img/mei-logo-03.jpg}
\end{wrapfigure}
Here we have how an image can be wrapped into the text without having surronding space, and takin advantage of the space to be disposed on the side, without breaking the text readability.

This approach also benefits from the fact that the text will be related implicitly to the image on its side, although the it should be referenced on the text anyway, otherwise, it should be consulting to perceive to which paragraph the image is related to.

Here is how we place an image as floating body.
Take in attention that the image is displayed on the next page, because there's no more room in this page.
\begin{figure}
\begin{center}
  \includegraphics[width=0.5\textwidth]{img/mei-logo-03.jpg}
\end{center}
\caption{caption}
\end{figure}



You can also use an image as an icon, eg.~\href{http://mei.di.uminho.pt}{\includegraphics[width=0.05\textwidth]{img/mei-logo-03.jpg}}, in the main tex.
Click on it to visit the website. It is also listed in the list of terms.
Another example of an item to appear in the term index: %\gls{um} (needs \Makeindex)
